\documentclass{article}

%\setlength{\voffset}{-2cm}

\usepackage[a4paper, landscape]{geometry}

\newcommand\cornTopic{Ordnungsrelationen} % today's topic
\usepackage{fullpage}
\usepackage{fancyhdr}
\usepackage[dvipsnames]{xcolor}
\usepackage{fontspec}
\usepackage{emoji}
\usepackage{tabularx,colortbl}
\usepackage{multirow}
\usepackage{booktabs}
\usepackage{makecell}
\usepackage{adjustbox}
\usepackage[ddmmyyyy]{datetime}
\usepackage{enumitem}
\usepackage{tcolorbox}

\setsansfont{Open Sauce Sans}
\setmonofont{Fira Code}
\renewcommand{\familydefault}{\sfdefault}

\definecolor{colBack}{HTML}{0c0c0c}
\definecolor{colFront}{HTML}{bbbbbb}
\definecolor{colCornGreen}{HTML}{3d721e}
\definecolor{colCornYellow}{HTML}{efc310}
\pagecolor{colBack}
\color{colFront}

\pagestyle{fancyplain}
\headheight 35pt
\lhead{\cornTopic}
\chead{
    \texttt{Evy's}\textbf{\Huge
        {\color{colCornYellow}CORN}{\color{colCornGreen}SHEETS}
        \emoji{corn}
    }
}
\rhead{\today{}\\\currenttime}
\headsep 1.5em

% colorbox for lemmas
\newtcolorbox{lemmaBox}[1]{
    colback=colBack,
    colframe=colCornGreen,
    coltext=colFront,
    title=Lemma #1
}

% colorbox for examples
\newtcolorbox{exampleBox}[1]{
    colback=colBack,
    colframe=colCornYellow,
    coltext=colFront,
    title=\textbf{\color{colBack}Beispiel: #1}
}

% colorbox for notes
\newtcolorbox{noteBox}{
    colback=colBack,
    colframe=gray,
    coltext=colFront,
    title=\textbf{\color{colBack}Notiz}
}

% for tables and such
\newcolumntype{L}[1]{>{\raggedright\arraybackslash}m{#1}}
\newcolumntype{C}[1]{>{\centering\arraybackslash}m{#1}} 
\newcolumntype{R}[1]{>{\raggedleft\arraybackslash}m{#1}}

\setlength{\extrarowheight}{10pt}

\begin{document}\large
\begin{adjustbox}{center}
	\setlength{\extrarowheight}{10pt}
	\begin{tabular}{L{4cm}C{2cm}C{4cm}C{3cm}C{2cm}C{4cm}C{3cm}C{3cm}}
		Ordnungsrelation                       & Reflixiv     & Antisymmetrisch & Asymmetrisch & Transitiv    & Alle Elemente sind vergleichbar & Nur unmittelbare Nachbern & Noethersch   \\
		[10pt]\midrule
		Partielle Ordnung / Halbordnung        & \emoji{corn} & \emoji{corn}    &              & \emoji{corn} &                                 &                           &              \\
		Quasiordnungen / Präordnung            & \emoji{corn} &                 &              & \emoji{corn} &                                 &                           &              \\
		Totale Quasiordnung / Präferenzordnung & \emoji{corn} &                 &              & \emoji{corn} & \emoji{corn}                    &                           &              \\
		Totale Ordnung / Lineare Ordnung       & \emoji{corn} & \emoji{corn}    &              & \emoji{corn} & \emoji{corn}                    &                           &              \\
		Striktordnung                          &              &                 & \emoji{corn} & \emoji{corn} &                                 &                           &              \\
		Nachbarschafts-ordnung                 &              &                 & \emoji{corn} & \emoji{corn} &                                 & \emoji{corn}              &              \\
		Noethersche Quasiordnung               & \emoji{corn} &                 &              & \emoji{corn} &                                 &                           & \emoji{corn} \\
	\end{tabular}
\end{adjustbox}
\end{document}