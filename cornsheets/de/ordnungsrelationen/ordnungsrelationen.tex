\documentclass{article}

\newcommand\cornTopic{Ordnungsrelationen} % today's topic
\usepackage{fullpage}
\usepackage{fancyhdr}
\usepackage[dvipsnames]{xcolor}
\usepackage{fontspec}
\usepackage{emoji}
\usepackage{tabularx,colortbl}
\usepackage{multirow}
\usepackage{booktabs}
\usepackage{makecell}
\usepackage{adjustbox}
\usepackage[ddmmyyyy]{datetime}
\usepackage{enumitem}
\usepackage{tcolorbox}

\setsansfont{Open Sauce Sans}
\setmonofont{Fira Code}
\renewcommand{\familydefault}{\sfdefault}

\definecolor{colBack}{HTML}{0c0c0c}
\definecolor{colFront}{HTML}{bbbbbb}
\definecolor{colCornGreen}{HTML}{3d721e}
\definecolor{colCornYellow}{HTML}{efc310}
\pagecolor{colBack}
\color{colFront}

\pagestyle{fancyplain}
\headheight 35pt
\lhead{\cornTopic}
\chead{
    \texttt{Evy's}\textbf{\Huge
        {\color{colCornYellow}CORN}{\color{colCornGreen}SHEETS}
        \emoji{corn}
    }
}
\rhead{\today{}\\\currenttime}
\headsep 1.5em

% colorbox for lemmas
\tcbset{lemmaTitle/.style={title={Lemma {#1}}}}

% for tables and such
\newcolumntype{L}[1]{>{\raggedright\arraybackslash}m{#1}}
\newcolumntype{C}[1]{>{\centering\arraybackslash}m{#1}} 
\newcolumntype{R}[1]{>{\raggedleft\arraybackslash}m{#1}}

\setlength{\extrarowheight}{10pt}

\begin{document}\large
\section*{Partielle Ordnung / Halbordnung}
\begin{defBox}[5.1]
	\begin{itemize}[label=\emoji{corn}]
		\item $\sqsubseteq$ ist \textbf{reflexiv}: {\color{colCornYellow} $\forall a \in A. a \sqsubseteq a$}
		\item $\sqsubseteq$ ist \textbf{antisymmetrisch}: {\color{colCornYellow} $\forall a_1, a_2 \in A. a_1 \sqsubseteq a_2 \land a_2 \sqsubseteq a_1 \Rightarrow a_1 = a_2$}
		\item $\sqsubseteq$ ist \textbf{transitiv}: {\color{colCornYellow} $\forall a_1, a_2, a_3 \in A. a_1 \sqsubseteq a_2 \land a_2 \sqsubseteq a_3 \Rightarrow a_1 \sqsubseteq a_3$}
	\end{itemize}
\end{defBox}
\begin{senBox}[5.1]
	\begin{center}
		$\leq$ ist eine partielle Ordnung auf $\mathbb{N}$
	\end{center}
\end{senBox}
\begin{exampleBox}[Partielle Ordnung / Halbordnung]
	\begin{itemize}[label=\emoji{corn}]
		\item $\subseteq$ auf $\mathfrak{P}(M)$ für eine beliebige Grundmenge M.
		\item Teilbarkeitsbeziehung $|$ auf $\mathbb{N}$.
		\item Teilzeichenreihenbeziehung auf $A^{*}$ definiert durch:
		      \[
			      w' \sqsubseteq w \Leftrightarrow_{df} \exists w_1, w_2 \in A^{*}. w_1\ w'\ w_2 = w
		      \]
		      (Beispiel: \texttt{w' = "sdf", w = "asdfg"})
	\end{itemize}
\end{exampleBox}

\section*{Quasiordnungen / Präordnung}
\begin{defBox}[5.2]
	\begin{itemize}[label=\emoji{corn}]
		\item $\sqsubsetsim$ ist \textbf{reflexiv}: {\color{colCornYellow} $\forall a \in A. a \sqsubsetsim a$}
		\item $\sqsubsetsim$ ist \textbf{transitiv}: {\color{colCornYellow} $\forall a_1, a_2, a_3 \in A. a_1 \sqsubsetsim a_2 \land a_2 \sqsubsetsim a_3 \Rightarrow a_1 \sqsubsetsim a_3$}
	\end{itemize}
\end{defBox}
\begin{exampleBox}[Quasiordnungen / Präordnung]
	\begin{itemize}[label=\emoji{corn}]
		\item ``kleiner oder gleich groß''-Beziehung bei Personen.
		\item Teilbarkeitsbeziehung $|$ auf $\mathbb{Z}$.
	\end{itemize}
\end{exampleBox}
\begin{noteBox}
	\begin{itemize}[label=\emoji{corn}]
		\item Eine Quasiordnung $\sqsubsetsim\ \subseteq A \times A$ induziert eine Äquivalenzrelation auf $A$ durch:
		      \[
			      a_1 \sim a_2 \Leftrightarrow_{df} a_1 \sqsubsetsim a_2 \land a_2 \sqsubsetsim a_1
		      \]
		\item $a_1 = a_2 \Rightarrow a_1 \sim a_2$ (\textbf{reflexiv})
		\item $\sqsubsetsim$ bildet eine partielle Ordnung auf $A \backslash \sim$
	\end{itemize}
\end{noteBox}

\section*{Totale Quasiordnung}
\begin{defBox}[totale Quasiordnung / Präferenzordnung]
	Eine Quasiordnung $\sqsubsetsim\ \subseteq A \times A$,
	in der \textbf{alle} Elemente vergleichbar sind,
	heißt \textit{totale Quasiordnung} oder auch \textit{Präferenzordnung}, d.h.:
	\[
		\forall a_1, a_2 \in A. a_1 \sqsubsetsim a_2 \lor a_2 \sqsubsetsim a_1
	\]
\end{defBox}
\begin{exampleBox}[totale Quasiordnung / Präferenzordnung]
	\begin{itemize}[label=\emoji{corn}]
		\item Personen nach ihrer Größe geordnet.
		\item ``Weniger mächtig''-Beziehung $\leqq$ auf Mengensystemen.
	\end{itemize}
\end{exampleBox}

\section*{Totale Ordnung}
\begin{defBox}[totale Ordnung / lineare Ordnung]
	Eine partielle Ordnung $\sqsubseteq\ \subseteq A \times A$,
	in der \textbf{alle} Elemente vergleichbar sind,
	heißt \textit{totale Ordnung} oder auch \textit{lineare Ordnung}, d.h.:
	\[
		\forall a_1, a_2 \in A. a_1 \sqsubseteq a_2 \lor a_2 \sqsubseteq a_1
	\]
\end{defBox}
\begin{exampleBox}[totale Ordnung / lineare Ordnung]
	\begin{itemize}[label=\emoji{corn}]
		\item $\leq$ auf $\mathbb{N}$.
		\item Lexikographische Ordnung auf $A^{*}$
	\end{itemize}
\end{exampleBox}

\section*{Striktordnungen}
\begin{defBox}[Striktordnungen]
	Zu einer gegebenen Quasiordnung $\sqsubsetsim$
	lässt sich die zugehörige \textit{Striktordnung}
	$\sqsubset$ definieren durch:
	\[
		a_1 \sqsubset a_2 \Leftrightarrow a_1 \sqsubsetsim a_2 \land a_1 \not \sim a_2
	\]
\end{defBox}
\begin{lemmaBox}[5.1]
	\begin{itemize}[label=\emoji{corn}]
		\item $\sqsubset$ ist \textbf{asymmetrisch}, d.h.: {\color{colCornYellow} $\forall a_1, a_2 \in A, a_1 \sqsubset a_2 \Rightarrow a_2 \not \sqsubset a_1$}
		\item $\sqsubset$ ist \textbf{transitiv}, d.h.: {\color{colCornYellow} $\forall a_1, a_2, a_3 \in A. a_1 \sqsubset a_2 \land a_2 \sqsubset a_3 \Rightarrow a_1 \sqsubset a_3$}
	\end{itemize}
\end{lemmaBox}
Folgerung: $\sqsubset$ ist \textbf{irreflexiv}, d.h.: $\forall a \in A. a \not \sqsubset a$

\section*{Partielle Ordnung aus Striktordnung}
\begin{defBox}[Partielle Ordnung aus Striktordnung]
	Zu einer gegebenen Striktordnung $\sqsubset$ lässt
	sich die zugehörige partielle \\
	Ordnung definieren durch:
	\[
		a_1 \sqsubseteq a_2 \Leftrightarrow_{df} a_1 \sqsubset a_2 \lor a_1 = a_2
	\]
\end{defBox}

\section*{Nachbarschaftsordnung}
\begin{defBox}
	Wird eine Striktordnung auf die unmittelbar benachbarten Abhängigkeiten \\
	reduziert, so entsteht die Nachbarschaftsordnung $\sqsubset_N$, definiert durch:
	\[
		a_1 \sqsubset_N a_2 \Leftrightarrow_{df} a_1 \sqsubset a_2 \land \not \exists a_3 \in A. a_1 \sqsubset a_3 \sqsubset a_2
	\]
\end{defBox}

\section*{Noethersche Quasiordnung}
\begin{defBox}
	Eine Quasiordnung $(M, \sqsubseteq)$ ist genau dann Noethersch,
	wenn es in M keine unendliche, echt absteigende Kette
	$x_0 \sqsupset x_1 \sqsupset x_2 \dots$ gibt.
\end{defBox}
\begin{exampleBox}{5.3 Noethersch partielle Ordnungen}
	\begin{itemize}[label=\emoji{corn}]
		\item $\leq$ auf \mathbb{N} ist Noethersch,
		      denn jede nichtleere Teilmenge enthält
		      sogar ein kleinstes Element.
		\item Die Teilzeichenreihenbeziehung auf $A^{*}$ ist Noethersch.
		\item $\subseteq$ ist Noethersch auf $\mathfrak{P}(M)$ für jede endliche Grundmenge M.
	\end{itemize}
\end{exampleBox}
\end{document}