\documentclass{article}

\newcommand\cornTopic{Mengen} % today's topic
\usepackage{fullpage}
\usepackage{fancyhdr}
\usepackage[dvipsnames]{xcolor}
\usepackage{fontspec}
\usepackage{emoji}
\usepackage{tabularx,colortbl}
\usepackage{multirow}
\usepackage{booktabs}
\usepackage{makecell}
\usepackage{adjustbox}
\usepackage[ddmmyyyy]{datetime}
\usepackage{enumitem}
\usepackage{tcolorbox}

\setsansfont{Open Sauce Sans}
\setmonofont{Fira Code}
\renewcommand{\familydefault}{\sfdefault}

\definecolor{colBack}{HTML}{0c0c0c}
\definecolor{colFront}{HTML}{bbbbbb}
\definecolor{colCornGreen}{HTML}{3d721e}
\definecolor{colCornYellow}{HTML}{efc310}
\pagecolor{colBack}
\color{colFront}

\pagestyle{fancyplain}
\headheight 35pt
\lhead{\cornTopic}
\chead{
    \texttt{Evy's}\textbf{\Huge
        {\color{colCornYellow}CORN}{\color{colCornGreen}SHEETS}
        \emoji{corn}
    }
}
\rhead{\today{}\\\currenttime}
\headsep 1.5em

% colorbox for lemmas
\tcbset{lemmaTitle/.style={title={Lemma {#1}}}}

% for tables and such
\newcolumntype{L}[1]{>{\raggedright\arraybackslash}m{#1}}
\newcolumntype{C}[1]{>{\centering\arraybackslash}m{#1}} 
\newcolumntype{R}[1]{>{\raggedleft\arraybackslash}m{#1}}

\setlength{\extrarowheight}{10pt}

\usepackage{amssymb}
\usepackage{amsmath}

\begin{document}\large
\section*{Aufzählung endlicher Mengen}
\begin{center}
    „Unter einer „Menge” verstehen wir jede Zusammenfassung M \\
    von bestimmten wohlunterschiedenen Objekten m unserer \\
    Anschauung oder unseres Denkens (welche die „Elemente” \\
    von M genannt werden) zu einem Ganzen.“
    \textit{Zitat: Georg Cantor}
\end{center}

\begin{exampleBox}{Mengen}
	\begin{itemize}[label=\emoji{corn}]
		\item $F = \{\clubsuit, \spadesuit, \heartsuit, \diamondsuit\}$
		\item $W = \{\mathsf{Mo}, \mathsf{Di}, \mathsf{Mi}, \mathsf{Do}, \mathsf{Fr}, \mathsf{Sa}, \mathsf{So}\}$
		\item $Z = \{2, 5, 123, -23, 666\}$
		\item Die Menge der natürlichen Zahlen $\mathbb{N}$ 
	\end{itemize}
\end{exampleBox}
\begin{noteBox}
    \begin{itemize}[label=\emoji{corn}]
        \item Reihenfolge der Aufzählung ist nicht relevant: $\{1,2,3\} = \{3,1,2\}$
        \item Elemente haben keine Häufigkeit: $\{1,1,1,2,2,3\} = \{1,2,3\}$
    \end{itemize}
\end{noteBox}
\end{document}