\documentclass{article}

\newcommand\cornTopic{Prädikatenlogik} % today's topic
\usepackage{fullpage}
\usepackage{fancyhdr}
\usepackage[dvipsnames]{xcolor}
\usepackage{fontspec}
\usepackage{emoji}
\usepackage{tabularx,colortbl}
\usepackage{multirow}
\usepackage{booktabs}
\usepackage{makecell}
\usepackage{adjustbox}
\usepackage[ddmmyyyy]{datetime}
\usepackage{enumitem}
\usepackage{tcolorbox}

\setsansfont{Open Sauce Sans}
\setmonofont{Fira Code}
\renewcommand{\familydefault}{\sfdefault}

\definecolor{colBack}{HTML}{0c0c0c}
\definecolor{colFront}{HTML}{bbbbbb}
\definecolor{colCornGreen}{HTML}{3d721e}
\definecolor{colCornYellow}{HTML}{efc310}
\pagecolor{colBack}
\color{colFront}

\pagestyle{fancyplain}
\headheight 35pt
\lhead{\cornTopic}
\chead{
    \texttt{Evy's}\textbf{\Huge
        {\color{colCornYellow}CORN}{\color{colCornGreen}SHEETS}
        \emoji{corn}
    }
}
\rhead{\today{}\\\currenttime}
\headsep 1.5em

% colorbox for lemmas
\newtcolorbox{lemmaBox}[1]{
    colback=colBack,
    colframe=colCornGreen,
    coltext=colFront,
    title=Lemma #1
}

% colorbox for examples
\newtcolorbox{exampleBox}[1]{
    colback=colBack,
    colframe=colCornYellow,
    coltext=colFront,
    title=\textbf{\color{colBack}Beispiel: #1}
}

% colorbox for notes
\newtcolorbox{noteBox}{
    colback=colBack,
    colframe=gray,
    coltext=colFront,
    title=\textbf{\color{colBack}Notiz}
}

% for tables and such
\newcolumntype{L}[1]{>{\raggedright\arraybackslash}m{#1}}
\newcolumntype{C}[1]{>{\centering\arraybackslash}m{#1}} 
\newcolumntype{R}[1]{>{\raggedleft\arraybackslash}m{#1}}

\setlength{\extrarowheight}{10pt}

\begin{document}
\large
\section*{Quantoren für All- und Existenzaussagen}

\begin{itemize}[label=\emoji{corn}]
	\item {\Large Allaussage:}
	      $\forall n. A(n)$

	      Diese ist genau dann wahr, wenn $A(n)$ für
	      alle Werte $n \in \mathbf{N}$ wahr ist. \\

	\item {\Large Existenzaussage:}
	      $\exists n. A(n)$

	      Diese ist genau dann wahr, wenn $A(n)$ für
	      mindestens einen Wert $n \in \mathbf{N}$ wahr ist.
\end{itemize}
\begin{lemmaBox}[2.2]
	\begin{itemize}[label=\emoji{corn}]
		\item $\neg (\forall x. A(x)) \equiv \exists x. \neg A(x)$
		\item $\neg (\exists x. A(x)) \equiv \forall x. \neg A(x)$
	\end{itemize}
\end{lemmaBox}


\section*{Prädikatenlogik über natürliche Zahlen}
\begin{center}
	\begin{tabular}{C{3cm}|C{4cm}|L{6cm}}
		Prädikat     & Definition                                                            & Bedeutung                                             \\
		[10pt]\midrule
		$n|m$        & $\exists k. n \cdot k = m$                                            & $n$ teilt $m$                                         \\
		[10pt]
		$ggT(n,m,x)$ & $x|n \land x|m \land \forall y. (y|n \land y|m) \Rightarrow y \leq x$ & Ist $x$ der größte gemeinsame Teiler von $n$ und $m$? \\
	\end{tabular}
\end{center}
\end{document}